\documentclass[a4paper]{article}

\usepackage[portuguese]{babel}
\usepackage[utf8]{inputenc}
\usepackage{indentfirst}
\usepackage{graphicx}
\usepackage{verbatim}
\usepackage{wrapfig}
\graphicspath{ {./img/} }

\begin{document}

\setlength{\textwidth}{16cm}
\setlength{\textheight}{22cm}

\title{\Huge\textbf{Adaptoid}\linebreak\linebreak\linebreak
\Large\textbf{Relatório Final}\linebreak\linebreak
\linebreak\linebreak
\includegraphics[scale=0.1]{feup-logo.png}\linebreak\linebreak
\linebreak\linebreak
\Large{Mestrado Integrado em Engenharia Informática e Computação} \linebreak\linebreak
\Large{Programação em Lógica}\linebreak
}

\author{\textbf{Grupo 04 : Adaptoid}\\ David Azevedo - 201405846 \\ João Ferreira - 201404332 \\\linebreak\linebreak \\
 \\ Faculdade de Engenharia da Universidade do Porto \\ Rua Roberto Frias, s\/n, 4200-465 Porto, Portugal \linebreak\linebreak\linebreak
\linebreak\linebreak\vspace{1cm}}
\date{Novembro de 2016}
\maketitle
\thispagestyle{empty}

%************************************************************************************************
%************************************************************************************************

\newpage

\section*{Resumo}
O problema abordado foi a implementação do jogo de tabuleiro "Adaptoid" em linguagem Prolog, o que nos representou como sendo um novo desafio, visto que, não estamos habituados a este paradigma de programação em lógica. Os objectivos do projecto incluem : permitir três modos de utilização (Humano/Humano, Humano/Computador, Computador/Computador), incluir dois níveis de jogo para o computador, interface adequada em modo de texto. Todos os objectivos foram cumpridos sendo que temos um jogo completo desenvolvido em Prolog. Desenvolver um jogo completo e funcional numa linguagem nova com a qual apenas tivemos contacto durante algumas semana não é fácil, pelo que o grupo no inicio encontrou algumas dificuldades. Após uma leitura extensiva dos slides dos docentes assim como informação disponível na web foi possível obter uma melhor percepção desta linguagem e de como desenvolver um jogo.
Concluindo, ambos os estudantes orgulham-se agora do resultado obtido e do método como tal foi alcançado, podemos também afirmar que o nosso conhecimento de Prolog aumentou consideravelmente.
\newpage

\tableofcontents

%************************************************************************************************
%************************************************************************************************

%*************************************************************************************************
%************************************************************************************************

\newpage

%%%%%%%%%%%%%%%%%%%%%%%%%%
\section{Introdução}

Descrever os objetivos e motivação do trabalho. Descrever num parágrafo breve a estrutura do relatório.
Este projecto foi proposto no âmbito da unidade curricular de Programação em Lógica do Mestrado Integrado de Engenharia Informática e de Computação da Faculdade de Engenharia da Universidade do Porto. Consiste na adaptação de um jogo de tabuleiro para dois jogadores na linguagem Prolog. O tema escolhido foi o "Adaptoid" um jogo relativamente simples de compreender e aprender a jogar mas que a quantidade de opções disponíveis ao jogador fazem com que o jogo tenha uma complexidade muito superior relativo ao que aparenta.
Este relatório tem a seguinte estrutura :
 \begin{itemize}
   \item Descrição do jogo, a sua história e regras.
   \item Implementação da lógica do jogo em Prolog, forma de representação do estado do tabuleiro e sua visualização, execução de movimentos, verificação do cumprimento das regras do jogo, determinação do final do jogo e cálculo das jogadas a realizar pelo computador.
   \item Módulo de interface com o utilizador em modo de texto.
   \item Conclucões.
   \item Bibliografia.
   \item Anexos.
 \end{itemize}



%%%%%%%%%%%%%%%%%%%%%%%%%%
\section{O Jogo Adaptoid}

\begin{wrapfigure}{r}{0.22\textwidth} 
    \centering
    \includegraphics[width=0.22\textwidth]{adaptoid}
    \caption{Imagem ilustratica de um "adaptoid"}
\end{wrapfigure}


Adaptoid é um jogo de tabuleiro para dois jogadores, constituído por um tabuleiro hexagonal, que contém (37 espaços), e por um conjunto de criaturas denominadas de “adaptoid”. Cabe a cada jogador evoluir o seu “adaptoid” adicionando membros, garras e pernas, ao corpo do adaptoid. Os membros são fatores decisivos, pois fazem variar o comportamento do “adaptoid”. As garras definem o dano e as pernas a capacidade de movimento. 

\begin{wrapfigure}{l}{0.20\textwidth}
    	\centering 
	\vspace{-10pt}
   	\includegraphics[width=0.20\textwidth]{adaptoidsDissecados}
    	\caption{Corpo e membros de um "adaptoid"}
\end{wrapfigure}

Cada turno divide-se em 3 fases distintas, sendo elas, movimento, crescimento e alimentação. Durante a fase de movimento o jogador pode mover um dos seus “adaptoids”, o número de espaços percorridos depende do número de pernas desse “adaptoid”. 

Não é possível mover o adaptoid através de espaços que estejam ocupados, apenas é possível mover em direção a espaços vazios sem obstáculos e mover para o topo de um “adaptoid” oposto, para iniciar a captura. Na fase de crescimento o jogador pode optar por um de dois casos possíveis, ou cria um novo corpo adjacente a um dos seus “adaptoids” existentes, ou então adiciona uma perna, ou garra, a um dos seus “adaptoids” existentes no tabuleiro. Na fase de alimentação, é verificada em cada peça do inimigo se ela está com fome, ou seja, o número  de espaços vazios à sua volta terá que ser igual ou maior ao número de membros do “adaptoid”. No caso de fome, a peça inimiga morre, é removida e é atribuído um ponto ao jogador. Durante a captura o “adaptoid ”com mais garras ganha e o “adaptoid ” derrotado é removido do tabuleiro. Em caso do número de garras dos “adaptoids” ser igual, ambos são removidos e cada jogador recebe um ponto. As peças de “adaptoids” mortos poderão ser novamente usadas. O jogo termina quando um jogador chega aos 5 pontos ou quando um dos jogadores ficar sem nenhum “adaptoid” no tabuleiro.


\begin{figure}[h]
\includegraphics[scale=1.0]{jogoDecorrer}
\caption{Estado de um jogo de adaptoid}
\centering
\end{figure}

%%%%%%%%%%%%%%%%%%%%%%%%%%
\section{Lógica do Jogo}

Descrever o projeto e implementação da lógica do jogo em Prolog, incluindo a forma de representação do estado do tabuleiro e sua visualização, execução de movimentos, verificação do cumprimento das regras do jogo, determinação do final do jogo e cálculo das jogadas a realizar pelo computador utilizando diversos níveis de jogo. Sugere-se a estruturação desta secção da seguinte forma:

\subsection{Representação do Estado do Jogo} Pode ser idêntico ao descrito no relatório intercalar.)

\subsection{Visualização do Tabuleiro} (Pode ser idêntico ao descrito no relatório intercalar.)

\subsection{Lista de Jogadas Válidas} Obtenção de uma lista de jogadas possíveis. Exemplo: \textit{valid\_moves(+Board, -ListOfMoves)}.

\subsection{Execução de Jogadas} Validação e execução de uma jogada num tabuleiro, obtendo o novo estado do jogo. Exemplo: \textit{move(+Move, +Board, -NewBoard)}.

\subsection{Avaliação do Tabuleiro} Avaliação do estado do jogo, que permitirá comparar a aplicação das diversas jogadas disponíveis. Exemplo: \textit{value(+Board, +Player, -Value)}.

\subsection{Final do Jogo} Verificação do fim do jogo, com identificação do vencedor. \\Exemplo: \textit{game\_over(+Board, -Winner)}.

\subsection{Jogada do Computador} Escolha da jogada a efetuar pelo computador, dependendo do nível de dificuldade. Por exemplo: \textit{choose\_move(+Level, +Board, -Move)}.


%%%%%%%%%%%%%%%%%%%%%%%%%%
\section{Interface com o Utilizador}

Descrever o módulo de interface com o utilizador em modo de texto.


%%%%%%%%%%%%%%%%%%%%%%%%%%
\section{Conclusões}
Que conclui deste projecto? Como poderia melhorar o trabalho desenvolvido?


\clearpage
%\addcontentsline{toc}{section}{Bibliografia}
%\renewcommand\refname{Bibliografia}
%\bibliographystyle{plain}
%\bibliography{myrefs}

\newpage
\appendix
\section{Nome do Anexo}
Código Prolog implementado devidamente comentado e outros elementos úteis que não sejam essenciais ao relatório.

\end{document}
