% This is LLNCS.DEM the demonstration file of
% the LaTeX macro package from Springer-Verlag
% for Lecture Notes in Computer Science,
% version 2.4 for LaTeX2e as of 16. April 2010
%
\documentclass{llncs}
%
\usepackage[portuguese]{babel}
\usepackage[utf8]{inputenc}
\usepackage{makeidx}  % allows for indexgeneration

%

\begin{document}

%
\frontmatter          % for the preliminaries
%
\pagestyle{headings}  % switches on printing of running heads
%
\title{Ano Escolar}
\subtitle{Resolução de Problemas de Desisão utilizando\\
Programação em Lógica com Restrições}
%
\titlerunning{Ano Escola}  % abbreviated title (for running head)
%                                     also used for the TOC unless
%                                     \toctitle is used
%
\author{Daviz Azevedo \and João Ferreira}
%
\authorrunning{Daviz Azevedo \and João Ferreira} % abbreviated author list (for running head)
%
\institute{Faculdade de Engenharia da Universidade do Porto\\
Rua Roberto Frias, sn, 4200-465 Porto, Portugal}

\maketitle              % typeset the title of the contribution

\begin{abstract} %



Deve contextualizar e resumir o trabalho, salientando o objetivo, o
método utilizado e fazendo referência aos principais resultados e à principal conclusão que
esses resultados permitem obter. \dots
\keywords{computational geometry, graph theory, Hamilton cycles}
\end{abstract}
%
\section{Introdução}
%
Descrição dos objetivos e motivação do trabalho, referência
sucinta ao problema em análise (idealmente, referência a outros trabalhos sobre o mesmo
problema e sua abordagem), e descrição sucinta da estrutura do resto do artigo.

\section{Descrição do Problema}
%
Descrever com detalhe o problema de otimização ou decisão em análise.

\section{Abordagem}
%
Descrever com detalhe o problema de otimização ou decisão em análise.

\subsection{Variáveis de Decisão}
Descrever as variáveis de decisão e os seus domínios.

\subsection{Restrições}
 Descrever as restrições rígidas e flexíveis do problema e a sua implementação utilizando o SICStus Prolog.

\subsection{ Função de Avaliação}
Descrever, quando for o caso, a forma de avaliar a solução obtida e a sua implementação utilizando o SICStus Prolog.

\subsection{Estratégia de Pesquisa}
Descrever a estratégia de etiquetagem
(labeling) utilizada ou implementada, nomeadamente no que diz respeito à ordenação
de variáveis e valores.

\section{Visualização da Solução}
Explicar os predicados que permitem visualizar a solução em modo de texto

\section{Resultados}
Demonstrar exemplos de aplicação em instâncias do problema com
diferentes complexidades e analisar os resultados obtidos. Devem ser utilizadas formas
convenientes para apresentação dos resultados (tabelas e/ou gráficos).

\section{Conclusões e Trabalho Futuro}
 Que conclusões retira deste projeto? O que mostram os resultados obtidos? Quais as vantagens e limitações da
solução proposta? Como poderia melhorar o trabalho desenvolvido?

\begin{figure}
\vspace{2.5cm}
\caption{This is the caption of the figure displaying a white eagle and
a white horse on a snow field}
\end{figure}

\begin{table}
\caption{This is the example table taken out of {\it The
\TeX{}book,} p.\,246}
\begin{center}
\begin{tabular}{r@{\quad}rl}
\hline
\multicolumn{1}{l}{\rule{0pt}{12pt}
                   Year}&\multicolumn{2}{l}{World population}\\[2pt]
\hline\rule{0pt}{12pt}
8000 B.C.  &     5,000,000& \\
  50 A.D.  &   200,000,000& \\
1650 A.D.  &   500,000,000& \\
1945 A.D.  & 2,300,000,000& \\
1980 A.D.  & 4,400,000,000& \\[2pt]
\hline
\end{tabular}
\end{center}
\end{table}

\begin{equation}
\begin{array}{rcl}
  \dot{x}&=&JH' (x)\\
  x(0)&=&x (T)
\end{array}
\end{equation}


\begin{proposition}
Assume $H'(0)=0$ and $ H(0)=0$. Set:
\begin{equation}
  \delta := \liminf_{x\to 0} 2 N (x) \left\|x\right\|^{-2}\ .
  \label{eq:one}
\end{equation}

If $\gamma < - \lambda < \delta$,
the solution $\overline{u}$ is non-zero:
\begin{equation}
  \overline{x} (t) \ne 0\ \ \ \forall t\ .
\end{equation}
\end{proposition}


\paragraph{Notes and Comments.}
The results in this section are a
refined version of \cite{clar:eke};
the minimality result of Proposition
14 was the first of its kind.

To understand the nontriviality conditions, such as the one in formula
(\ref{eq:four}), one may think of a one-parameter family
$x_{T}$, $T\in \left(2\pi\omega^{-1}, 2\pi b_{\infty}^{-1}\right)$
of periodic solutions, $x_{T} (0) = x_{T} (T)$,
with $x_{T}$ going away to infinity when $T\to 2\pi \omega^{-1}$,
which is the period of the linearized system at 0.

%
% ---- Bibliography ----
%
\begin{thebibliography}{5}
%
\bibitem {url} 
SWI-Prolog,
\url{http://www.swi-prolog.org}
\bibitem {url} 
SICStus-Prolog,
\url{https://sicstus.sics.se}


\end{thebibliography}
\clearpage

\section*{Anexo}
\subsection*{Código fonte}

Bla Bla


\end{document}
